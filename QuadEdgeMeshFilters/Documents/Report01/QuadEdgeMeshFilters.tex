\documentclass{InsightArticle}

\usepackage[dvips]{graphicx}
\usepackage{color}
\usepackage{listings}

\definecolor{listcomment}{rgb}{0.0,0.5,0.0}
\definecolor{listkeyword}{rgb}{0.0,0.0,0.5}
\definecolor{listnumbers}{gray}{0.65}
\definecolor{listlightgray}{gray}{0.955}
\definecolor{listwhite}{gray}{1.0}

\makeatletter
\newcommand\ackname{Acknowledgements}
\if@titlepage
  \newenvironment{acknowledgements}{%
      \titlepage
      \null\vfil
      \@beginparpenalty\@lowpenalty
      \begin{center}%
        \bfseries \ackname
        \@endparpenalty\@M
      \end{center}}%
     {\par\vfil\null\endtitlepage}
\else
  \newenvironment{acknowledgements}{%
      \if@twocolumn
        \section*{\abstractname}%
      \else
        \small
        \begin{center}%
          {\bfseries \ackname\vspace{-.5em}\vspace{\z@}}%
        \end{center}%
        \quotation
      \fi}
      {\if@twocolumn\else\endquotation\fi}
\fi
\makeatother

%%%%%%%%%%%%%%%%%%%%%%%%%%%%%%%%%%%%%%%%%%%%%%%%%%%%%%%%%%%%%%%%%%
%
%  hyperref should be the last package to be loaded.
%
%%%%%%%%%%%%%%%%%%%%%%%%%%%%%%%%%%%%%%%%%%%%%%%%%%%%%%%%%%%%%%%%%%
\usepackage[dvips,
bookmarks,
bookmarksopen,
backref,
colorlinks,linkcolor={blue},citecolor={blue},urlcolor={blue},
]{hyperref}


\title{QuadEdgeMesh Shift and Window Scalars Filters}


% 
% NOTE: This is the last number of the "handle" URL that 
% The Insight Journal assigns to your paper as part of the
% submission process. Please replace the number "1338" with
% the actual handle number that you get assigned.
%
\newcommand{\IJhandlerIDnumber}{1338}


\release{1.00}

\author{Wen Li$^{1,2}$, Vincent A. Magnotta$^{1,2,3}$}
\authoraddress{$^{1}$Department of Radiology, The University of Iowa, Iowa City, IA 52242\\
                          $^{2}$Department of Biomedical Engineering, The University of Iowa, Iowa City, IA 52242\\
                          $^{3}$Department of Psychiatry, The University of Iowa, Iowa City, IA 52242}

\begin{document}


%
% Add hyperlink to the web location and license of the paper.
% The argument of this command is the handler identifier given
% by the Insight Journal to this paper.
% 
\IJhandlefooter{\IJhandlerIDnumber}


\ifpdf
\else
   %
   % Commands for including Graphics when using latex
   % 
   \DeclareGraphicsExtensions{.eps,.jpg,.gif,.tiff,.bmp,.png}
   \DeclareGraphicsRule{.jpg}{eps}{.jpg.bb}{`convert #1 eps:-}
   \DeclareGraphicsRule{.gif}{eps}{.gif.bb}{`convert #1 eps:-}
   \DeclareGraphicsRule{.tiff}{eps}{.tiff.bb}{`convert #1 eps:-}
   \DeclareGraphicsRule{.bmp}{eps}{.bmp.bb}{`convert #1 eps:-}
   \DeclareGraphicsRule{.png}{eps}{.png.bb}{`convert #1 eps:-}
\fi


\maketitle


\ifhtml
\chapter*{Front Matter\label{front}}
\fi


\begin{abstract}
\noindent
This document describes a couple of QuadEdgeMesh Scalar Filters which can
shift and window the scalar values on input mesh by linear transformations.

This paper is accompanied with the source code, input data, parameters and
output data that we used for validating the algorithm described in this paper.
This adheres to the fundamental principle that scientific publications must
facilitate \textbf{reproducibility} of the reported results.
\end{abstract}

\begin{acknowledgements}
This work was funded in part by NIH/NINDS award NS050568.
\end{acknowledgements}

\tableofcontents

\section{Introduction}

itkShiftScalarsQuadEdgeMeshFilter:

Each point's scalar value is shifted by $m_Shift$ and scaled by $m_Scale$.

If the scalar values are moved out of the range of OutputPixelType,
they are set to be NonpositiveMin and Max of the pixel type.

itkWindowScalarsQuadEdgeMeshFilter:

Scalar values inside of $[m_WindowMinimum, m_WindowMaximum]$ are mapped into
$[m_OutputMinimum, m_OutputMaximum]$. Scalar values lower than $m_WindowMinimum$ 
are set to be equal to $m_WindowMinimum$ and scalar values higher than $m_WindowMaximum$
are set to be equal to $m_WindowMaximum$ before scale mapping.


\section{How to Build}

This contribution includes

\begin{itemize}
\item QuadEdgeMesh Shift Scalar Filter and Window Scalar Filter
\item Tests for the filters
\item All the LaTeX source files of this paper
\end{itemize}

The testing code is in Testing directory as

\begin{itemize}
\item itkShiftScalarQuadEdgeMeshFilterTest.cxx
\item itkWindowScalarQuadEdgeMeshFilterTest.cxx
\end{itemize}

\subsection{Building Executables and Tests}

In order to build the whole, it is enough to configure the directory with
CMake. As usual, an out-of-source build is the recommended method.

In a Linux environment it should be enough to do the following:

\begin{itemize}
\item \code{ccmake  SOURCE\_DIRECTORY}
\item \code{make}
\item \code{ctest}
\end{itemize}

Where SOURCE\_DIRECTORY is the directory where you have expanded the source
code that accompanies this paper.

This will configure the project, build the executables, and run the tests and
examples. 


\subsection{Building this Report}

In order to build this report you can do

\begin{itemize}
\item \code{ccmake SOURCE\_DIRECTORY}
\item Turn ON the CMake variable
\begin{itemize}
\item \code{BUILD\_REPORTS}
\end{itemize}
\item \code{make}
\end{itemize}

This should produce a PDF file in the binary directory, under the subdirectory
\code{Documents/Report01}.

\section{How to Use the Filter}

\end{document}


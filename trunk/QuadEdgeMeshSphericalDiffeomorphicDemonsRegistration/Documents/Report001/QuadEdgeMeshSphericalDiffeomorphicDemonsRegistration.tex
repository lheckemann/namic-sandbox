\documentclass{InsightArticle}

\usepackage[dvips]{graphicx}

%%%%%%%%%%%%%%%%%%%%%%%%%%%%%%%%%%%%%%%%%%%%%%%%%%%%%%%%%%%%%%%%%%
%
%  hyperref should be the last package to be loaded.
%
%%%%%%%%%%%%%%%%%%%%%%%%%%%%%%%%%%%%%%%%%%%%%%%%%%%%%%%%%%%%%%%%%%
\usepackage[dvips,
bookmarks,
bookmarksopen,
backref,
colorlinks,linkcolor={blue},citecolor={blue},urlcolor={blue},
]{hyperref}


\title{Rotational Registration of Spherical Surfaces Represented as QuadEdge Meshes}


% 
% NOTE: This is the last number of the "handle" URL that 
% The Insight Journal assigns to your paper as part of the
% submission process. Please replace the number "1338" with
% the actual handle number that you get assigned.
%
\newcommand{\IJhandlerIDnumber}{1338}


\release{1.00}

\author{Michel Audette$^{1}$, Luis Ibanez$^{1}$, Thomas Yeo$^{1}$, Polina Goland$^{2}$}
\authoraddress{$^{1}$Kitware Inc., Clifton Park, NY\\
               $^{2}$CSAIL MIT, Boston, MA}

\begin{document}


%
% Add hyperlink to the web location and license of the paper.
% The argument of this command is the handler identifier given
% by the Insight Journal to this paper.
% 
\IJhandlefooter{\IJhandlerIDnumber}


\ifpdf
\else
   %
   % Commands for including Graphics when using latex
   % 
   \DeclareGraphicsExtensions{.eps,.jpg,.gif,.tiff,.bmp,.png}
   \DeclareGraphicsRule{.jpg}{eps}{.jpg.bb}{`convert #1 eps:-}
   \DeclareGraphicsRule{.gif}{eps}{.gif.bb}{`convert #1 eps:-}
   \DeclareGraphicsRule{.tiff}{eps}{.tiff.bb}{`convert #1 eps:-}
   \DeclareGraphicsRule{.bmp}{eps}{.bmp.bb}{`convert #1 eps:-}
   \DeclareGraphicsRule{.png}{eps}{.png.bb}{`convert #1 eps:-}
\fi


\maketitle


\ifhtml
\chapter*{Front Matter\label{front}}
\fi


\begin{abstract}
\noindent
This document describes a contribution to the Insight Toolkit intended to
support the process of registering two Meshes.  The methods included here are
restricted to Meshes with a Spherical geometry and topology, and with scalar
values associated to their nodes.

This paper is accompanied with the source code, input data, parameters and
output data that we used for validating the algorithm described in this paper.
This adheres to the fundamental principle that scientific publications must
facilitate \textbf{reproducibility} of the reported results.
\end{abstract}

\tableofcontents

\section{Introduction}

The Insight Toolkit already provides methods for registering

\begin{itemize}
\item Image to Image
\item PointSet to Image
\item PointSet to Pointset
\end{itemize}

but it lacks methods for registering one Mesh versus another Mesh.

In this paper we contribute new classes that can be used for performing
registration between two spherical meshes, although not all of of the classes
in this contribution are restricted to be used on spherical meshes.

\section{Overview}

The design of these classes follows very closely the one of the Image
Registration Framework in ITK.  In particular, we have the usual components

\begin{itemize}
\item Optimizer
\item Metric
\item Transform
\item Interpolator
\end{itemize}

and we have the two objects to be registered, in this case Meshes instead of
Images. The two main components that must be providded in order to support Mesh
registration are Iterpolators and Metrics.

The following diagram indicates the hierarchy of classes that are included 
in this contribution.

\begin{figure}
\center
\includegraphics[width=0.7\textwidth]{ClassHierarchy.pdf}
\itkcaption[Registration Class Hierarchy]{Hierarchy of new Registration Classes.}
\label{fig:ClassHierarchy}
\end{figure}

\end{document}

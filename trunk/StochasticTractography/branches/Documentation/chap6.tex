\chapter{Conclusion and Further Work}
In this project we have implemented a system for analyzing white matter architecture from DTI images.  The multithreaded implementation of the system allows parallel sampling of fiber tracts, reducing computation time on multi-processor systems.  To facilitate the algorithm's use in clinical studies, an easy to use graphical interface was created for the 3D Slicer visualization program. Furthermore, we demonstrate the system's characteristics through an analysis fibers originating in the right internal capsule.  Finally, we have presented the outline for an ongoing group study which compares frontal lobe fibers in schizophrenia and control groups using the stochastic tractography system developed in this thesis.

The stochastic tractography system's weighted least squares estimation of the log tensor model parameters is but one possible method to estimate the diffusion tensor parameters for DWI data.  Potentially more accurate techniques of parameter estimaton have been described by Koay et al. \cite{koay06}.  The design of the stochastic tractography system is not tightly coupled and allows the tensor parameter estimation engine to be replaced.  Ideally, however, the DTI image and estimated B0 image should be calculated before the stochastic tractography system is run.  The stochastic tractography system should be able to take this DTI image and B0 image as input in lieu of the DWI image.  This allows the researcher to use any arbitrary tensor model parameter estimation engine and additionally would make stochastic tractography faster as the tensor parameters would not need to be calculated at runtime.  The modifications required to implement this feature is relatively minor, but the user interface must be designed thoughtfully to minimize the added complexity of these new options.

%Discuss finding a better prior
Although DTI is able to detect changes in anisotropy, DTI data alone is insufficient to determine the causes of changes in anisotropy.  Reduced anisotropy can stem from a decoherence in a fiber bundle's structure, losses in myelin, and other abnormalities.  Kubicki et al. \cite{kubickiNI05} attempted to clarify the cause of changes in anisotropy by comparing information about nerve fiber integrity that are inferred from DTI and MTR.  MTR is an MR scanning technique that provides information regarding the distribution of myelin in the brain. Anisotropy changes that are detectable by both MTR and DTI suggest alterations to the myelin.  Changes that are detectable only by DTI suggest decoherence in the fiber's structure.  The algorithm can be further extended by incorporating information from MTR measurements \cite{kubickiNI05} into the Bayesian framework.


%Refine the algorithm
%	-currently the algorithm terminates when the generated fiber encounters a region of %anisotropy below some set value
%		-in reality, fibers may exist that pass through areas of low anisotropy
%		-this is possible in regions with many crossing fibers
%			-perhaps it would be possible to terminate instead on the overall magnitude of %diffusion.  If total diffusion is above/below a certain threshold than end diffusion.

%Once this data has been obtained the connectivity map will be generated for all subjects.  The connectivity maps will be compared to determine whether there are differences in the connectivity distributions of the regions that were determined by Kubicki et al. to differ either under DTI or MTR measurements, between the patients and the control groups.  The regions that differed under fractional anisotropy (FA), a summary measure of DTI, include the fornix, the corpus callosum, the  cingulum bundle, the superior occipito-frontal fasciculus, the internal capsule, the right inferior occipito-frontal fasciculus and the left arcuate fasciculus.  The regions that differed under MTR were the corpus callosum, fornix, right internal capsule, superior occipito-frontal fasciculus and the right posterior cingulum bundle.

\documentclass{article}

\author{Tri M. Ngo}
\title{Probabilisitic Tractography and Its Applications in Neurological Studies}

\bibliographystyle{plain}

\usepackage{fullpage}

\usepackage{graphicx}
\usepackage{amsmath}
\usepackage{amsfonts}
\usepackage{amssymb}

\begin{document}
\maketitle

\begin{abstract}
Abstract
\end{abstract}

\section{Methods}
%Also decribe rationale for decisions.

%Scanning method
%  type of aquisition (i.e. Line scanning diffusion imaging)
%  machine type, characteristics
%  In which plane were the images acquired, etc.
%  Parameters of acquisition, b value, etc
%  Other images that were acquired.
%Describe subjects
%  Age, characteristics
  
%ROIs
%  how were they defined
%    segmented by an expert, Gudrun, with reference to the FA image
%    left and right IC's (more details on how they were defined)
%   An roi in prefrontal cortex (details on how this was defined)
%    Other ROIs used by the streamline tractography system
%    These voxels were used as seeds.

%Describe Streamline Method

%Describe Tractography Method
%  General Outline
%  Stopping Criteria
%  Prior on tract bending
  
%Analysis Method
%  ANOVA
%  T-Tests
\subsection{Data Acquisition}
Diffusion weighted images (DWI) and anatomical magnetic resonance images (MRI) were obtained from 22 controls and 19 patients diagnosed with chronic schizophrenia (mean duration of illness=16 years). The age range for subjects was from 23 to 55, and controls were group matched to patients on age, handedness, parental socio-economic status, and IQ. 

\subsection{Region of Interest Definition}
For all subjects, the left and the right internal capsules were segmented by an expert using Fractional Anisotropy (FA) images obtained from the DWI images as a reference. Additionally, a second, large region of interest (ROI) was placed within the prefrontal cortex.  Every voxel in the internal capsule segmentation was used as seeds for stochastic tractography.  The tracts of interests are defined as tracts which start in the internal capsule and eventually pass throught the prefrontal cortex ROI.

\subsection{Preprocessing}
%generating the white matter masks
%algorithm and steps used

\subsection{Stochastic Tractography}
%need to place in past tense, algorithm already implemented
%Remove comparisons with Behrens
%cite thesis
The stochastic tractography algorithm used in this study is based on the approach described by Friman's \cite{frimanTMI06} with some modifications to the stopping criteria.

%elaborate on the prior
%give a more outline feel
A fiber tract is modeled as a sequence of unit vectors.  The orientation of these unit vectors is determined by sampling a posterior fiber orientation distribution which is dependent on the local diffusion data as well as the orientation of the unit vector in the previous step.  The posterior distribution is the normalized product of the prior likelihood of the fiber orientation and the likelihood of that fiber orientation given the local diffusion data.
%show image of posterior distribution

For the observation model, the algorithm uses a subset of the tensor model which is called the constrained diffusion model.  In this model, the two smallest eigenvectors of diffusion tensor are equal, constraining the shape of the diffusion tensor to be linearly anisotropic.  The constrained model rules out the possibility of nonlinear, or non-cylindrical anisotropic diffusion distributions.  Deviations from linearly anisotropic diffusion distributions are captured as uncertainty in the fiber orientation.  The constrained model is combined with a Gaussian DWI noise model to obtain a fiber orientation likelihood function.  The parameters for the constrained model are derived from a weighted least squares estimation of the parameters for the log tensor model.

The orientation of each vector depends only on the previous vector.  This dependency is formulated in the prior on the fiber orientation.  Prior knowledge about the regularity of the fiber tract can encoded in this prior probability.  The prior also serves to prevent the fiber from backtracking, since the likelihood distribution alone is axially symmetric.

In streamline tractography, tract generation terminated when an encountered voxel's fiber orientation is uncertain due to low FA.  However, since FA is a statistic which is analyzed for group differences, it should not be used in tractography as that may bias the results.  Since the stochastic tractography algorithm explicitly takes into account this uncertainty with an increase in the spatial variance of sampled fibers, this termination criterion is unneccesary and contradictory with the goals of stochastic tractography, which is to enable sampling of tracts in regions of uncertainty.  Thus our version of stochastic tractography terminates tracking based on the posterior probability that a fiber tract exists within the current voxel.  The posterior probability that a fiber tract exists in a given voxel can be obtained by performing a soft segmentation of white matter on an anatomical image co-registered with the DWI data.  Alternatively, the soft segmentation can also be performed on the B0 image of the DWI data set, thus eliminating the need for additional data.  While this may seem equivalent to using an anisotropy threshold criterion, since white matter generally has higher anisotropy than gray matter, it does not exclude regions of white matter which have low anisotropy due to crossing fibers.  This criteria should enable the algorithm to detect more tracts than under the anisotropy termination criteria.

\subsection{Statistical Analysis}
ANOVA with side as a within subject factor, group as a between subject factor, and age as a covariate revealed a group effect for both the connectivity ratio (P=0.052) and the mean FA (P=0.048). In addition, these two measures were not correlated with each other. In contrast, the same ANOVA test for the streamline tractography generated fibers did not reveal group effects for FA (P=0.198). Additionally, using the streamline tractography method, we found that the number of fibers, as well as their length, was strongly correlated with FA.
  Connectivity Image
  Tract-averaged FA
  Frontal Lobe Fiber Ratio Calculation

\section{Acknowledgements}
\section{References}
\bibliography{main}

\end{document}

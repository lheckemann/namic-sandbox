\documentclass[a4paper,10pt]{article}
\usepackage[utf8x]{inputenc}
\usepackage{amssymb, amsmath, amsfonts}

% Title Page
\title{Real Spherical Harmonic basis}
\author{Luke Bloy}

\newcommand{\abs}[1]{\ensuremath{    \left| #1 \right| }}
\renewcommand\Re{\operatorname{\mathfrak{Re}}}
\renewcommand\Im{\operatorname{\mathfrak{Im}}}

\begin{document}
\maketitle

\begin{abstract}
The purpose of this section is to describe the basis implemented in itkRealSymSphericalHarmonicBasis.h and to descibe its relation to existing RSH basis in literature.
\end{abstract}

A number of high angular resolution diffuision imaging (HARDI) data models consist of real-valued antipodally symmetric functions defined on the sphere. The real spherical harmonics provide a natural functional basis for for functions of this type. Beacuse the spherical harmonics ($Y_l^m$) of odd order ($l$) are not symmetric they can be saftely removed from basis set.

We use the normalized real spherical harmonics inorder to maintain orthonomality. The basic definition of our basis set is as follows:
\begin{equation}
\label{eq:RSHdef}
R_l^m(\theta,\phi) = \begin{cases}
         Y_l^m(\theta,\phi)               & m = 0 \: \text{,} \: l \: \text{even}\\
	\sqrt{2} \Re{Y_l^m(\theta,\phi)}  & m < 0 \: \text{,} \: l \: \text{even}\\
        \sqrt{2} \Im{Y_l^m(\theta,\phi)}  & m > 0 \: \text{,} \: l \: \text{even}\\        
\end{cases}
\end{equation}

Often there are operators that have been defined in the complex spherical harmonics ($Y_l^m$), such as rotation. For this reason it is also often useful to  express $R_l^m$ directly interms of the $Y_l^m$s functions. Allowing us to write down a change of basis operation between the two functional basis sets directly. To achieve this we use the following identity for the complex congugate of $Y_l^m$.
\begin{equation*}
\label{eq:SHcomplexConjugate}
\overline{Y_l^m} = (-1)^m Y_l^{-m}
\end{equation*}
Using this identity we cas express $\Re{Y_l^m}$ and $\Im{Y_l^m}$ as follows:
\begin{align*}
\Re{Y_l^m} &= \frac{1}{2} \left( Y_l^m + \overline{Y_l^m} \right) = \frac{1}{2} \left( Y_l^m + (-1)^m Y_l^{-m} \right) \\
\Im{Y_l^m} &= \frac{1}{2 i} \left( Y_l^m - \overline{Y_l^m} \right) = \frac{i}{2} \left( - Y_l^m + (-1)^{m} Y_l^{-m} \right)
\end{align*}
This yeilds the following expression of equation \ref{eq:RSHdef}:
\begin{equation}
R_l^m(\theta,\phi) = \begin{cases}
         Y_l^m(\theta,\phi)               & m = 0 \: \text{,} \: l \: \text{even}\\
	\frac{\sqrt{2}}{2} \left( Y_l^m + (-1)^m Y_l^{-m} \right)  & m < 0 \: \text{,} \: l \: \text{even}\\
        \frac{\sqrt{2}i}{2} \left( - Y_l^m + (-1)^{m} Y_l^{-m} \right)  & m > 0 \: \text{,} \: l \: \text{even}\\        
\end{cases}
\end{equation}

In our implementation we use the definition of the complex spherical harmonics without the Condon-Shortley phase (http://mathworld.wolfram.com/Condon-ShortleyPhase.html) which is included in the definition of the legendre polynomials ( $P_l^m$ ) we use. 
\begin{equation*}
Y_l^m(\theta,\phi) = \sqrt{\frac{(2 l +1)}{4 \pi}} \sqrt{\frac{(l-m)!}{(l+m)!}} P_l^m (\cos \theta) e^{i m \phi}  
\end{equation*}

This definition yeilds the RSH basis as implemented in itkRealSymSphericalHarmonicBasis.h.
\begin{equation}
R_l^m(\theta,\phi) = \begin{cases}
        \sqrt{\frac{(2 l +1)}{4 \pi}} P_l^0 (\cos \theta)             & m = 0 \: \text{,} \: l \: \text{even}\\
	\sqrt{2} \sqrt{\frac{(2 l +1)}{4 \pi} \frac{(l-m)!}{(l+m)!}} P_l^m (\cos \theta) \cos(m \phi)    & m < 0 \: \text{,} \: l \: \text{even}\\
        \sqrt{2} \sqrt{\frac{(2 l +1)}{4 \pi} \frac{(l-m)!}{(l+m)!}} P_l^m (\cos \theta) \sin(m \phi)    & m > 0 \: \text{,} \, l \: \text{even}\\        
\end{cases}
\end{equation}

It is important to note thet many implementations of the assosiated legendre polynomials do not support negative $m$. For this reason, we use the following identity when computing $P_l^m$ for $m < 0$:
\begin{equation*}
P_l^{-m} = (-1)^m \frac{(l+m)!}{(l-m)!} P_l^{m}
\end{equation*}

\end{document}          

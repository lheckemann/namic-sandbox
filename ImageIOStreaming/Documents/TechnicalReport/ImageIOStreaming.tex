%
% Complete documentation on the extended LaTeX markup used for Insight
% documentation is available in ``Documenting Insight'', which is part
% of the standard documentation for Insight.  It may be found online
% at:
%
%     http://www.itk.org/

\documentclass{InsightArticle}

\usepackage[dvips]{graphicx}

%%%%%%%%%%%%%%%%%%%%%%%%%%%%%%%%%%%%%%%%%%%%%%%%%%%%%%%%%%%%%%%%%%
%
%  hyperref should be the last package to be loaded.
%
%%%%%%%%%%%%%%%%%%%%%%%%%%%%%%%%%%%%%%%%%%%%%%%%%%%%%%%%%%%%%%%%%%
\usepackage[dvips,
bookmarks,
bookmarksopen,
backref,
colorlinks,linkcolor={blue},citecolor={blue},urlcolor={blue},
]{hyperref}


%  This is a template for Papers to the Insight Journal. 
%  It is comparable to a technical report format.
\title{Implementing IO Streaming in ITK Image Readers}

% Increment the release number whenever significant changes are made.
% The author and/or editor can define 'significant' however they like.
\release{1.00}

% At minimum, give your name and an email address.  You can include a
% snail-mail address if you like.
\author{Hans Johnson$^{1}$, Douglas Alan$^{2}$ and Luis Ib\'{a}\~{n}ez$^{3}$}
\authoraddress{$^{1}$University of Iowa\\
               $^{2}$Harward University, IIC\\
               $^{3}$Kitware Inc.}

\begin{document}


\ifpdf
\else
   %
   % Commands for including Graphics when using latex
   % 
   \DeclareGraphicsExtensions{.eps,.jpg,.gif,.tiff,.bmp,.png}
   \DeclareGraphicsRule{.jpg}{eps}{.jpg.bb}{`convert #1 eps:-}
   \DeclareGraphicsRule{.gif}{eps}{.gif.bb}{`convert #1 eps:-}
   \DeclareGraphicsRule{.tiff}{eps}{.tiff.bb}{`convert #1 eps:-}
   \DeclareGraphicsRule{.bmp}{eps}{.bmp.bb}{`convert #1 eps:-}
   \DeclareGraphicsRule{.png}{eps}{.png.bb}{`convert #1 eps:-}
\fi


\maketitle


\ifhtml
\chapter*{Front Matter\label{front}}
\fi


% The abstract should be a paragraph or two long, and describe the
% scope of the document.
\begin{abstract}
\noindent
This document describes an improvement to the Image IO infrastructure of the
Insight Toolkit ITK. The toolkit as a whole was designed and implemented to
support streaming data accross filters. The purpose fo this functionality was
to allowing performing image processing in large images without having to load
the entire image into memory. Instead, the processing can be performed by
loading smaller pieces of the image and processing every piece independently.

Unfortunately, the implementation of streaming was currently not usable due to
the lack of streaming support at the level of the readers. This report
describes the modifications that must be applied to the Image IO infrastructure
in order to support streaming at read time. 
\end{abstract}

\tableofcontents

\section{Introduction}

One of the initial design requirements of ITK was to be able to process images
larger than the actual RAM of the computer in which the processing was
performed. This requirement was motivated by the need of processing the data
from the Visible Human datasets. In order to satisfy this requirement, the
pipeline architecture of ITK included the concept of \emph{RequestedRegion},
that was negotiated filters placed in contiguous locations of the pipeline.

Typically, the requested region was expected to be smaller than the total size
of the image, and even smaller than the portion of the image that was loaded
into memory at that point. By requesting regions smaller than the total size of
the image, filters in the pipeline were empowered to process large images
without needing to have the entire image in memory at a given time. The image was
partitioned in sub-regions and each one of them was processed sequentially.

This streaming architecture, is currently fully implemented in the ITK
pipeline, but suffers from the limitation that most of the image file readers
were not supporting it. As a consequence, when the request for a smaller porting
of the full image arrived to the the ultimate data source: the reader, it was 
addressed by disregarding the request and loading the full image into memory.

Honoring the request for a smaller region of the image, however, needed only a set
of relatively minor modifications in the Image IO infrastructure of the toolkit, 
as well as customized modifications in every one of the ImageIO classes.

This paper describes the modifications required in the basic infrastructure,
and provide examples on how to modify some of the ImageIO classes. In
particular the YAFF, MetaImage, FITS and Nifty image IO classes.



\section{Streaming Architecture}

\subsection{What was missing in the ImageFileReader}

\subsection{YAFF Reader}
\subsection{FITS Reader}
\subsection{Nifti Reader}
\subsection{MetaImage Reader}

\section{Tests and Examples}


%%%%%%%%%%%%%%%%%%%%%%%%%%%%%%%%%%%%%%%%%
%
%  Insert the bibliography using BibTeX
%
%%%%%%%%%%%%%%%%%%%%%%%%%%%%%%%%%%%%%%%%%

\bibliographystyle{plain}
\bibliography{InsightJournal}


\end{document}


%
% Complete documentation on the extended LaTeX markup used for Insight
% documentation is available in ``Documenting Insight'', which is part
% of the standard documentation for Insight.  It may be found online
% at:
%
%     http://www.itk.org/

\documentclass{InsightArticle}

\usepackage[dvips]{graphicx}

%%%%%%%%%%%%%%%%%%%%%%%%%%%%%%%%%%%%%%%%%%%%%%%%%%%%%%%%%%%%%%%%%%
%
%  hyperref should be the last package to be loaded.
%
%%%%%%%%%%%%%%%%%%%%%%%%%%%%%%%%%%%%%%%%%%%%%%%%%%%%%%%%%%%%%%%%%%
\usepackage[dvips,
bookmarks,
bookmarksopen,
backref,
colorlinks,linkcolor={blue},citecolor={blue},urlcolor={blue},
]{hyperref}


%  This is a template for Papers to the Insight Journal. 
%  It is comparable to a technical report format.
\title{Reader/Writer for Analyze Object Maps for ITK}

% Increment the release number whenever significant changes are made.
% The author and/or editor can define 'significant' however they like.
\release{1.00}

% At minimum, give your name and an email address.  You can include a
% snail-mail address if you like.
\author{Jeffrey Hawley, Hans Johnson }
\authoraddress{University of Iowa}

\begin{document}


\ifpdf
\else
   %
   % Commands for including Graphics when using latex
   % 
   \DeclareGraphicsExtensions{.eps,.jpg,.gif,.tiff,.bmp,.png}
   \DeclareGraphicsRule{.jpg}{eps}{.jpg.bb}{`convert #1 eps:-}
   \DeclareGraphicsRule{.gif}{eps}{.gif.bb}{`convert #1 eps:-}
   \DeclareGraphicsRule{.tiff}{eps}{.tiff.bb}{`convert #1 eps:-}
   \DeclareGraphicsRule{.bmp}{eps}{.bmp.bb}{`convert #1 eps:-}
   \DeclareGraphicsRule{.png}{eps}{.png.bb}{`convert #1 eps:-}
\fi


\maketitle


\ifhtml
\chapter*{Front Matter\label{front}}
\fi


% The abstract should be a paragraph or two long, and describe the
% scope of the document.
\begin{abstract}
\noindent
This document describes an addition to the Image IO Library of the Insight Toolkit (ITK).
ITK has been able to read in Analyze image files but not the object maps that correspond
to the images.  Without the object maps, regions of interest were not being shown.
These regions of interest are helpful in designating locations, giving a name to those
locations, as well as colors and other information, which is useful if you are trying to tell
someone where to look in an image.  This report describes the reason why this code should be
added into ITK and how to use the code.
\end{abstract}

\tableofcontents

\section{Anaylze Object Maps}
An Analyze Object Map is an image that is plotted with numbers from 0-255 that correspond to an Analyze Object(or, as will be referred to in this paper as, Analyze Object Entry).  Each Analyze Object Entry
is a specific Region of Interest that a user has specified. With these Regions of Interest the user can put a name to the Region, a specific color and be able to communicate to other people what areas to look at or know which Regions they are talking about.  Regions of Interest could be differentiating a part of the brain, marking a location where someone thinks a tumor is or many other ways to group a specific area to a name and color.  The Analyze Object Maps can be shown by themselves or overlaid of another image, such as the original image that was the basis of the Analyze Object Map, to show which areas were highlighted.  
  
\section{Files}
There are two utility files (objectmap and objectentry) and the main IO file (itkAnalyzeObjectLabelMapImageIO) and the corresponding
IO factory file (itkAnalyzeObjectLabelMapImageIOFactory).  

The IO file takes care of reading and writing out an object file.  Whenever
you read or write an object file you have at the very beginning information about the object map.  The very first data is the version
of the object map, next you have the x,y,z size of the object map, the number of object entries and finally for version 7 (the newest
version) number of volumes.  After you have the beginning information there is the information about each object file, that means that
there is going to have to be a vector of all of the entries.  

That is why there is a utility file called objectentry that stores the
information for one specific entry, if you make a vector of the file then you have all of the object entries for an object map.  There
are also some other functions in the objectentry that will read or write the information for the object entry.  

Finally, after you have all of the entries you have a run length encoding.  The run length encoding corresponds to an image of the numbers of the entries starting from 0 and going all the way up to the number of object entries.  

The objectmap file has functions that take care of doing things to an object map once it is created - such as converting the objectmap to an RGB Image. 

The following bullets are the instructions on how to incorprate the files into itk and the functions inside of each file.

\begin{itemize}
\item How to include the files into ITK.

The way to include all of the files we have written will go like this:

You will have to place itkAnalyzeObjectLabelMapImageIO.h, itkAnalyzeObjectLabelMapImageIO.cxx, itkAnalyzeObjectLabelMapImageIOFactory.h, itkAnalyzeObjectLabelMapImageIOFactory.cxx into the source of ITK-$>$Code-$>$IO.  While in the source of ITK-$>$Code-$>$IO you will want to change the CMakeLists file so that SET{...} has itkAnalyzeObjectLabelMapImageIOFactory.cxx and itkAnalyzeObjectLabelMapImageIO.cxx in it.  Then you will need to modify itkImageIOFactory.cxx, you will want place \#inlcude "itkAnalyzeObjectLabelMapImageIOFactory.h", with the includes, then add in the function RegisterBuiltInFactories the following code: ObjectFactoryBase::RegisterFactory( AnalyzeObjectLabelMapImageIOFactory::New() );

Then you will have to place the contents of the folder called AnalyzeObjectMap into the source of ITK-$>$Code-$>$Common.  You will also have to modfify the CMakeLists file so that SET{...} has itkAnalyzeObjectEntry.cxx in it.

\item itkAnalyzeObjectLabelMapIO

Functions:
\begin{itemize}
\item CanReadFile

This checks to see if the filename ends with an obj extension.  If it does then return true that this class can read the file.

\item ReadImageInformation

This reads in the header information.  It first reads in the verision of the object map, then the x,y,z sizes, then the number of objects and for verision 7 object maps the number of volumes.  Then the function reads in each object entry and stores them into a vector of object entries.

\item Read

This will read in and decipher the run length encoded image.

\item CanWriteFile

This checks to see if the filename ends with an obj extension.  If it does then return true that this class can write the file.

\item WriteImageInformation

This writes out the header information.  It first writes out the verision of the object map (which is verisoin 7 right now), the x,y,z sizes, the number of objects and for verision 7 object maps the number of volumes.  Then the function writes out each object entry to the file.  The object entry will most likly take up the most room of the file.  If an image was created without any object entries then the writer will write out 256 object entries.

\item Write

This will comprese the image by run length encoding the image and then write the run length encoding to the specified file.

\item CanStreamRead

This will return false right now because the reader can not stream data.  This is something that will be worked on in the future.

\end{itemize}

\item itkAnalyzeObjectEntry

Functions:
\begin{itemize}
      
      \item Copy
       
      This function will copy all of the ivars except the name of the entry.  The reason why the function does not copy the name, is that each object entry should have a unique name.

      \item These next functions are get/set macros that will get/set most of the ivars.\\ \\

         getName/setName, getDisplayFlag/setDisplayFlag, getCopyFlag/setCopyFlag, getMirrorFlag/setMirrorFlag, getStatusFlag/setStatusFlag, getNeighborsUsedFlag/setNeighborsUsedFlag, getShades/setShades, getStartRed/setStartRed, getStartGreen/setStartGreen, getStartBlue/setStartBlue, getEndRed/setEndRed, getEndGreen/setEndGreen, getEndBlue/setEndBlue, getXRotation/setXRotation, getXRotationIncrement/setXRotationIncrement, getYRotation/setYRotation, getYRotationIncrement/setYRotationIncrement, getZRotation/setZRotation, getZRotationIncrement/setZRotationIncrement, getXTranslation/setXTranslation, getXTranslation/setXTranslation, getYTranslation/setYTranslation, getYTranslationIncrement/setYTranslationIncrement, getZTranslation/setZTranslation, getZTranslation/setZTranslation, getXCenter/setXCenter, getYCenter/setYCenter, getZCenter/setZCenter, getMinimumXValue/setMinimumXValue, getMinimumYValue/setMinimumYValue, getMinimumZValue/setMinimumZValue, getMaximumXValue/setMaximumXValue, getMaximumYValue/setMaximumYValue, getMaximumZValue/setMaximumZValue, getOpacity/setOpacity,   GetOpacityThickness/SetOpacityThickness, GetBlendFactor/SetBlendFactor\\ \\

  \item Print

    This function will print out all of the ivars out to any file that the user wants.  This is mostly used for debugging purposes.

  \item ReadFromFilePointer
    
    This function will read in all of the ivars from a file location that is passed into it.

  \item SwapObjectEndedness

    This function will change the object endedness if the computer is a little endian machine, since the object maps are written in big endian.

  \item Write
  
    This function will write out all of the ivars to a file location that is passed into it.
\end{itemize}

\item itkAnalyzeObjectMap

  Functions:
\begin{itemize}
\item AnalyzeObjectMap operator=

This function is the assignment operator, which copies a new object map from the right side to the AnalyzeObjectMap variable on the left side.

\item GetAnalyzeObjectEntryArrayPointer

This will return a pointer to the vector of object entries that an object map has.

\item GetNumberOfObjects/SetNumberOfObjects

This will get/set the number of object entries that an object map has.

\item PickOneEntry

The user will input the number of the Object entry that they would like to pick.  Then the function will create a new object map that will re returned.  Then the function will go through the original object map image and get rid of all of the other object numbers and change the number of the object entry the user specified to one.  Then the new image will be outputted to the new Object map.  After that the object entry from the original object map will be copied over to the new object map's vector of object entries. 

\item ObjectMapToRGBImage

This will convert the object map into an RGB Image based on the end red, end green and end blue specified for each object entry.  That means that the function will have to go through the object map image, get the value at each pixel, find the object entry that corresponds to the value of the pixel and then pull out the end red, end green and end blue and set that as the pixel color for the RGB Image.  Then that RGB Image will be returned.

\item AddObjectEntryBasedOnImagePixel

This will go through an image that the user inputs, find the specific pixel value the user inputs and then create an object map at the location that it finds the specific pixel value.  The user will also have the option of inputing the red, green and blue they want the object map to be.

\item AddAnalyzeObjectEntry

This will just add an object entry to the end of the vector of object entries that an object map has.

\item DeleteAnalyzeObjectEntry

This will delete an object entry that a user specifies.  The function will go through the image and delete the number that corresponds to the object entry.  Then the function will move all of the object entry numbers above the object entry that was deleted down one number.  Then the function will move all of the object entries above the object entry that was deleted in the vector one number down.

\item FindObjectEntry

This function will find an object entry based on the name that the user inputs.  If the function finds the object entry then it will return the number of the vector of the object entry.  If the function does not find the object entry then the function will return -1.

\item PlaceObjectMapEntriesIntoMetaData

This function will place the object entries into the meta data so that the object map can be moved around just like a normal image.  This function is normally called in the functions that are in this class.

\item GetObjectEntry

This function will return the smart pointer of the object entry the user inputs.

\item ImageToObjectMap

This function will take an image and make it into an object map.  If there is data for object entries in the meta data then extract that data.  Then take the pixel container of the image and place it into the object map's pixel container.
\end{itemize}
\end{itemize}


\section{Examples}

This section provides some examples on how to use the programs.
\begin{itemize}
\item Example 1 

This example is for reading in any file, then creating a new object map, then add entries to the object map and finally writing the new object map out to file.


Command:

\textbf{CreatingObjects.exe  [source\_directory/Data/Input]/2dtest.nii creatingObjects.obj}\\

\small \begin{verbatim}
#include "itkImageFileReader.h"
#include "itkImageFileWriter.h"
#include "itkImage.h"
#include "itkAnalyzeObjectMap.h"

int main( int argc, char ** argv )
{
  typedef unsigned char       InputPixelType;
  
  typedef unsigned char       OutputPixelType;
  
  const   unsigned int        Dimension = 3;
  
  typedef itk::Image< InputPixelType,  Dimension >    InputImageType;
  
  typedef itk::Image< OutputPixelType, Dimension >    OutputImageType;
  
  typedef itk::ImageFileReader< InputImageType  >  ReaderType;
  
  typedef itk::ImageFileWriter< OutputImageType >  WriterType;
  ReaderType::Pointer reader = ReaderType::New();
  WriterType::Pointer writer = WriterType::New();
  ...
  reader->SetFileName(NiftiFile);
  ...
  reader->Update();
  ...
  itk::AnalyzeObjectMap<InputImageType>::Pointer CreateObjectMap = itk::AnalyzeObjectMap
  <InputImageType>::New();
  
  ...
  
  //Add one entry to the object map named "You Can Delete Me", this entry corresponds to 1
  if you do a pickOneEntry
  CreateObjectMap->AddAnalyzeObject("You Can Delete Me");

  //Add another two entries that will be based on the image that is passed into 
  //the function, also, the intensity that you would like searched for, the name of the entry
  and then finally the RGB values
  //you would like the entry to have for the regions that are found.

  //This entry corrsponds to 2 if you do a pickOneEntry
  CreateObjectMap->AddObjectEntryBasedOnImagePixel(reader->GetOutput(), 200, "Square", 250,
  0, 0);

  //This entry corrsponds to 3 if you do a pickOneEntry
  CreateObjectMap->AddObjectEntryBasedOnImagePixel(reader->GetOutput(), 128, "Circle", 0,
  250, 0);

  //This entry corrsponds to 4 if you do a pickOneEntry
  CreateObjectMap->AddObjectEntryBasedOnImagePixel(reader->GetOutput(), 45,  "SquareTwo",
  0, 0, 250);

  //Then anoter entry is added, this entry corrsponds to 5 if you do a pickOneEntry
  CreateObjectMap->AddAnalyzeObject("Nothing In Here");

  //The entry that was just added is deleted
  CreateObjectMap->DeleteAnalyzeObject("Nothing In Here");
  
  ...
  writer->SetInput(CreateObjectMapTwo);
  ...
  
  writer->Update();
  ...
}
\end{verbatim} \normalsize

\item Example 2

This is a simple example for reading in an Analyze object map and
then showing how to use vtk to display the object map. 


Command:

\textbf{DisplayingObjectMaps.exe creatingObjects.obj}

\small \begin{verbatim}
#include "itkImage.h"
#include "itkRGBPixel.h"
#include "itkImageFileReader.h"
#include "itkImageFileWriter.h"
#include "itkAnalyzeObjectMap.h"

//#include "itkImageToVTKImageFilter.h"
//#include "vtkRenderer.h"
//#include "vtkRenderWindowInteractor.h"
//#include <vtkImageViewer2.h>

typedef unsigned char PixelType;
const unsigned int Dimension = 3;
typedef itk::Image< PixelType, Dimension > ImageType;
typedef itk::RGBPixel<PixelType> RGBPixelType;
typedef itk::Image< RGBPixelType, Dimension > RGBImageType;

typedef itk::ImageFileReader< ImageType > ReaderType;

//typedef itk::ImageToVTKImageFilter<RGBImageType> ConnectorType;

int main(int argc, char * argv [] )
{
  ReaderType::Pointer reader  = ReaderType::New();
  
  //The input should be an Anaylze Object Map file
  reader->SetFileName( DisplayImage );
  ...
  
  reader->Update();
  
  //This will convert the output of the reader into an object map
  itk::AnalyzeObjectMap<ImageType, RGBImageType>::Pointer Objectmap = itk::AnalyzeObjectMap
  <ImageType, RGBImageType>::New();
  Objectmap->ImageToObjectMap(reader->GetOutput());
  
  //If you have vtk and itkApplications installed then you can uncomment this out to display
  //an object map to the screen.  Otherwise you can see how to display an object map using 
  vtk.
  
  //Set the input, to the connector connecting itk with vtk, with an RGB image of the 
  specific
  //colors that corresponds to each entry in the object map.
  
  //vtkRenderWindowInteractor *windowInteractor = vtkRenderWindowInteractor::New();
  //ConnectorType::Pointer connector= ConnectorType::New();
  //connector->SetInput( Objectmap->ObjectMapToRGBImage() );
  
  //connector->Update();
  
  //Display a two dimensional view of the object map that was read in
  //vtkImageViewer2 * twodimage = vtkImageViewer2::New();
  
  //const int SliceNumber = 0;
  //twodimage->SetInput(connector->GetOutput());
  //twodimage->SetSlice(SliceNumber);
  //twodimage->SetSliceOrientationToXY();
  
  //Set the background of the renderer to a grayish color so that it is easier to see
  //the outline of the object map since it is usually black
  // twodimage->GetRenderer()->SetBackground(0.4392, 0.5020, 0.5647);
  // twodimage->SetupInteractor(windowInteractor);
   
  // twodimage->Render();
  // windowInteractor->Start();
	...
}
\end{verbatim} \normalsize
                                                                               
\item Example 3

This example is for reading in any file, then creating a new object map, then add entries to the object map, pick one entry and then write out the new object map with the one entry only.

Command:

\textbf{PickOneObjectEntry.exe [source\_directory/Data/Input]/2dtest.nii circle.obj}

\small \begin{verbatim}
#include "itkImageFileReader.h"
#include "itkImageFileWriter.h"
#include "itkImage.h"
#include "itkAnalyzeObjectMap.h"

int main( int argc, char ** argv )
{
  ...
  //Now we bring in a nifti file that Hans and Jeffrey created, 
  //the image as two squares and a circle in it of different intensity values.
  reader->SetFileName(NiftiFile);
  ...
  reader->Update();
  ...
  
  itk::AnalyzeObjectMap<InputImageType>::Pointer CreateObjectMap = itk::AnalyzeObjectMap
  <InputImageType>::New();
  
  //Add another two entries that will be based on the image that is passed into 
  //the function, also, the intensity that you would like searched for, 
  //the name of the entry and then finally the RGB values
  //you would like the entry to have for the regions that are found.
  CreateObjectMap->AddObjectEntryBasedOnImagePixel(reader->GetOutput(), 200, "Square", 250,
  0, 0);
  CreateObjectMap->AddObjectEntryBasedOnImagePixel(reader->GetOutput(), 128, "Circle", 0,
  250,0);
  CreateObjectMap->AddObjectEntryBasedOnImagePixel(reader->GetOutput(), 45,  "SquareTwo", 
  0, 0, 250);
  
  //Pick the circle entry and have it put into CreateObjectMap two.  These means
  //that there is only one entry in CreateObjectMapTwo and the image has also
  //been taken care of.
  itk::AnalyzeObjectMap<InputImageType>::Pointer CreateObjectMapTwo = CreateObjectMap->
  PickOneEntry(2);
  
  //Place all of the entries into the meta data so that the entries can be written
  //out to an object file.
  CreateObjectMapTwo->PlaceObjectMapEntriesIntoMetaData();
  
  //Now write out an object file
  writer->SetInput(CreateObjectMapTwo);
  writer->SetFileName(CreatingObject);
  ...
  writer->Update();
  ...
}
\end{verbatim} \normalsize

\end{itemize}

\section{Testing}

Right now we have one test that does all of the testing.  The filename for the test is AnalyzeObjectMapTest.cxx.  The file takes in seven arguments.
Each argument corresponds to a certain name of another file.  

For the first argument it should be the name of test.obj and that file is a 3-Dimensional object file that Hans created.  The file is read in, then converted into an object map and finally an RGB image is created from the object map.  This was just to make sure that everything worked without any errors.  

After that the output of the reader is sent to a writer to write out the exact copy of the argument but with a different file name.  That is where the second argument comes in.  Once the file is written out then both files are opened up and compared to make sure they have the exact same bit by bit data.  

After that the third argument that corresponds to the name of a nifti file that Hans and I created which is the same image seen in example 1 is read in.  With that nifti file read in we create a new object map with some extra entries and other entries based on the nifti file that was read in.  That new object map is written out a file as an object file that corresponds to the fourth argument read in.  That file that was just written out is then read back in where the reader output is then converted into an object map and finally an RGB image is created from the object map.  

Then from that object map one entry is picked and new object map is created with just the one entry in it.  This object map with the one entry is written to file that corresponds with the fifth argument.  The file that was just written out is read back in and then converted into an object map and finally an RGB image is created from the object map.

From there, the sitxh argument corresponds to a name for a four dimensional object name that will be written out.  A four dimensional image is created and then sent to the writer and then is read back in.

Finally, the seventh argument corresponds to a name for a one dimensional object name that will be written out.  A one dimensional image is created and then sent to the writer and then is read back in.

\section{Future Work}

We will make the writer and reader streamable in the future.


%%%%%%%%%%%%%%%%%%%%%%%%%%%%%%%%%%%%%%%%%
%
%  Insert the bibliography using BibTeX
%
%%%%%%%%%%%%%%%%%%%%%%%%%%%%%%%%%%%%%%%%%

\bibliographystyle{plain}
\bibliography{InsightJournal}


\end{document}


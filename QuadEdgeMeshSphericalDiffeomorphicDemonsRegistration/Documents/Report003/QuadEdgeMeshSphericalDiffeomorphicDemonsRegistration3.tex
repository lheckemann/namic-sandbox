\documentclass{InsightArticle}

\usepackage[dvips]{graphicx}
\usepackage{color}
\usepackage{listings}

\definecolor{listcomment}{rgb}{0.0,0.5,0.0}
\definecolor{listkeyword}{rgb}{0.0,0.0,0.5}
\definecolor{listnumbers}{gray}{0.65}
\definecolor{listlightgray}{gray}{0.955}
\definecolor{listwhite}{gray}{1.0}


%%%%%%%%%%%%%%%%%%%%%%%%%%%%%%%%%%%%%%%%%%%%%%%%%%%%%%%%%%%%%%%%%%
%
%  hyperref should be the last package to be loaded.
%
%%%%%%%%%%%%%%%%%%%%%%%%%%%%%%%%%%%%%%%%%%%%%%%%%%%%%%%%%%%%%%%%%%
\usepackage[dvips,
bookmarks,
bookmarksopen,
backref,
colorlinks,linkcolor={blue},citecolor={blue},urlcolor={blue},
]{hyperref}


\title{Assign Scalars Mesh Filter}


% 
% NOTE: This is the last number of the "handle" URL that 
% The Insight Journal assigns to your paper as part of the
% submission process. Please replace the number "1338" with
% the actual handle number that you get assigned.
%
\newcommand{\IJhandlerIDnumber}{3117}

\lstset{frame = tb,
       framerule = 0.25pt,
       float,
       fontadjust,
       backgroundcolor={\color{listlightgray}},
       basicstyle = {\ttfamily\footnotesize},
       keywordstyle = {\ttfamily\color{listkeyword}\textbf},
       identifierstyle = {\ttfamily},
       commentstyle = {\ttfamily\color{listcomment}\textit},
       stringstyle = {\ttfamily},
       showstringspaces = false,
       showtabs = false,
       numbers = left,
       numbersep = 6pt,
       numberstyle={\ttfamily\color{listnumbers}},
       tabsize = 2,
       language=[ANSI]C++,
       floatplacement=!h
       }



\release{1.00}

\author{Wen Li$^{1}$, Vincent Magnotta$^{1}$}
\authoraddress{$^{1}$University of Iowa}

\begin{document}


%
% Add hyperlink to the web location and license of the paper.
% The argument of this command is the handler identifier given
% by the Insight Journal to this paper.
% 
\IJhandlefooter{\IJhandlerIDnumber}


\ifpdf
\else
   %
   % Commands for including Graphics when using latex
   % 
   \DeclareGraphicsExtensions{.eps,.jpg,.gif,.tiff,.bmp,.png}
   \DeclareGraphicsRule{.jpg}{eps}{.jpg.bb}{`convert #1 eps:-}
   \DeclareGraphicsRule{.gif}{eps}{.gif.bb}{`convert #1 eps:-}
   \DeclareGraphicsRule{.tiff}{eps}{.tiff.bb}{`convert #1 eps:-}
   \DeclareGraphicsRule{.bmp}{eps}{.bmp.bb}{`convert #1 eps:-}
   \DeclareGraphicsRule{.png}{eps}{.png.bb}{`convert #1 eps:-}
\fi


\maketitle


\ifhtml
\chapter*{Front Matter\label{front}}
\fi


\begin{abstract}
 
This documents is about the filter itkAssignScalarValuesQuadEdgeMeshFilter.  It
takes two meshes, one as an input and the other as a source and assign scalar
values of the nodes on one mesh to another.  Both of the two meshes should have
the same number of nodes.

This paper is accompanied with the source code, input data, parameters and
output data that we used for validating the algorithm described in this paper.
This adheres to the fundamental principle that scientific publications must
facilitate \textbf{reproducibility} of the reported results.
\end{abstract}

\tableofcontents

\section{Introduction}


\section{How to Build}

This contribution includes

\begin{itemize}
\item Assign Scalars filter
\item Tests for the filter
\item Examples on how to use the filter
\item All the LaTeX source files of this paper
\end{itemize}

The source code is in Source directory and named as 

\begin{itemize}
\item itkAssignScalarValuesQuadEdgeMeshFilter.h
\item itkAssignScalarValuesQuadEdgeMeshFilter.cxx
\end{itemize}


The testing code is in Testing directory as

\begin{itemize}
\item itkAssignScalarValuesQuadEdgeMeshFilterTest1.cxx
\end{itemize}

The example about how to use it is in Example directory as

\begin{itemize}
\item RigidAndDemonsRegistration.cxx
\end{itemize}

\subsection{Building Executables and Tests}

In order to build the whole, it is enough to configure the directory with
CMake. As usual, an out-of-source build is the recommended method.

In a Linux environment it should be enough to do the following:

\begin{itemize}
\item \code{ccmake  SOURCE\_DIRECTORY}
\item \code{make}
\item \code{ctest}
\end{itemize}

Where SOURCE\_DIRECTORY is the directory where you have expanded the source
code that accompanies this paper.

This will configure the project, build the executables, and run the tests and
examples. 


\subsection{Building this Report}

In order to build this report you can do

\begin{itemize}
\item \code{ccmake SOURCE\_DIRECTORY}
\item Turn ON the CMake variable
\begin{itemize}
\item \code{BUILD\_REPORTS}
\end{itemize}
\item \code{make}
\end{itemize}

This should produce a PDF file in the binary directory, under the subdirectory
\code{Documents/Report003}.

\section{How to Use the Filter}

This section illustrates the minimum operations required for running 

%%%%%%%%%%%%%%%%%%%%%%%%%%%%%%%%%%%%%%%%%
%
%  Insert the bibliography using BibTeX
%
%%%%%%%%%%%%%%%%%%%%%%%%%%%%%%%%%%%%%%%%%

\bibliographystyle{plain}
\bibliography{InsightJournal}



\end{document}
